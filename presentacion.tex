 
% Para una visualizacion correcta, generar el PDF
% Ni el DVI ni el PS se visualizan bien

% Elegir el estilo que se desee, hay cientos en la red

\documentclass{beamer}
\setbeamertemplate{navigation symbols}{}
\usepackage{beamerthemeshadow}
%\usepackage[galician]{babel}
\usepackage[spanish]{babel}
\usepackage[utf8]{inputenc}
%\usepackage[latin1]{inputenc}


% \usetheme{Warsaw}
% \usetheme{Rochester}
% \usetheme{bars}
% \usetheme{Bergen}
% \usetheme{Boadilla}
% \usetheme{Boxes}
% \usetheme{classic}
% \usetheme{Madrid}
% \usetheme{Pittsburgh}
% \usetheme{Antibes}
% \usetheme{Montpellier}
% \usetheme{Berkeley}
% \usetheme{PaloAlto}
% \usetheme{Goettingen}
% \usetheme{Marburg}
% \usetheme{Hannover}
%\usetheme{JuanLesPins}
% \usetheme{Berlin}
% \usetheme{Ilmenau}
% \usetheme{Dresden}
% \usetheme{Luebeck}
%\usetheme{Darmstadt}
% \usetheme{Frankfurt}
% \usetheme{Singapore}
% \usetheme{Szeged}
%\usetheme{Copenhagen}
% \usetheme{lined}
% \usetheme{Malmoe}
% \usetheme{Tree}
% \usetheme{shadow}
% \usetheme{sidebar}
% \usetheme{split}
% \usetheme{default}
%\setbeamercovered{transparent}

%%%%%%%%%% Choose color scheme%%%%%%%%
% \usecolortheme{default}
% \usecolortheme{sidebartab}
% \usecolortheme{albatross}
%\usecolortheme{beetle}
%\usecolortheme{crane}
% \usecolortheme{dolphin}
% \usecolortheme{dove}
% \usecolortheme{fly}
% \usecolortheme{seagull}
% \usecolortheme{lily}
% \usecolortheme{orchid}
% \usecolortheme{seahorse}
% \usecolortheme{rose}
% \usecolortheme{whale}
% \usecolortheme{structure}
%%%%%%%%%%%%%%%%%%%%%%%%%%%%%%%%%%%

%\useoutertheme{shadow}
%\useoutertheme{infolines}
%\useoutertheme{split}

\hypersetup{pdfpagemode=FullScreen}

\begin{document}
\title{Visualización de observacións en SIX de escritorio}
\subtitle{Grao en Enxeñaría Informática \\
Universidade de Santiago de Compostela}  
\author{Rubén Mosquera Varela}
\institute{Director: José Ramón Ríos Viqueira\\Codirector: Manuel Antonio Regueiro Seoane}
\date{22 de Xullo de 2015} 

\begin{frame}
\titlepage
\end{frame}

\begin{frame}
\frametitle{Táboa de contidos}\tableofcontents
\end{frame} 

\section{Motivación e obxectivos} 
%\subsection{Ejemplo de subsección}
\begin{frame}
\frametitle{Motivación} 
%Cada pantalla tiene su título.
\end{frame}
%
%\subsection{Ejemplo de lista}
%
%\begin{frame}
%\frametitle{Lista no numerada}
%\begin{itemize}
%\item una  
%\item dos 
%\item tres 
%\item cuatro
%\end{itemize} 
%\end{frame}
%
%\begin{frame}
%\frametitle{Lista con pausa}
%\begin{itemize}
%\item número uno \pause 
%\item número dos \pause 
%\item número tres \pause 
%\item número cuatro
%\end{itemize} 
%\end{frame}
%
%\subsection{Otro ejemplo de lista}
%\begin{frame}
%\frametitle{Lista numerada}
%\begin{enumerate}
%\item una  
%\item dos 
%\item tres 
%\item cuatro
%\end{enumerate}
%\end{frame}
%
\section{Demostración}
%\subsection{Tablas}
%
%\begin{frame}
%\frametitle{Tablas}
%\begin{tabular}{|c|l|r|} \hline
%\textbf{Centrado} & \textbf{Izquierda} & \textbf{Derecha} \\ \hline
%AAAA  & 1000 & aaaa \\ \hline
%BB    & 20   & bb \\ \hline
%\end{tabular}
%\end{frame}
%
%\begin{frame}
%\frametitle{Tabla con pausa}
%\begin{tabular}{c c c}
%A & B & C \\ \pause 
%1 & 2 & 3 \\  \pause 
%A & B & C \\ 
%\end{tabular} 
%\end{frame}
%
\section{Conclusións}
%\subsection{Bloques}
%
\begin{frame}
\frametitle{Conclusións}
%
%\begin{block}{Bloque normal}
%Texto del bloque normal
%\end{block}
%
%\begin{exampleblock}{Bloque de ejemplo}
%Texto del bloque ejemplo
%\end{exampleblock}
%
%\begin{alertblock}{Bloque de alerta}
%Texto del bloque alerta
%\end{alertblock}
\end{frame}
%
%\section{Sección 4}
%\subsection{Pantalla dividida}
%
%\begin{frame}
%\frametitle{Pantalla dividida}
%\begin{columns}
%\begin{column}{5cm}
%\begin{itemize}
%\item una lista
%\item de puntos 
%\item mas una tabla 
%\end{itemize}
%\end{column}
%\begin{column}{5cm}
%\begin{tabular}{|c|c|c|} \hline
%\textbf{Mes} & \textbf{Día} & \textbf{Hora} \\ \hline
%Enero   & 10 & 15:30 \\ \hline
%Febrero & 20 & 20:00 \\ \hline
%\end{tabular}
%\end{column}
%\end{columns}
%\end{frame}
%
%\subsection{Figuras} 
%\begin{frame}
%\frametitle{Incluir figuras}
%\begin{figure}
%\includegraphics[scale=0.3]{images/logo_usc.eps} 
%\caption{Logo de la USC}
%\end{figure}
%\end{frame}
%
%\subsection{Listas con figuras y pausas} 
%
%\begin{frame}
%\frametitle{Listas con figuras y pausas}
%\begin{columns}
%\begin{column}{4cm}
%\begin{itemize}
%\item<1-> Una
%\item<3-> Dos
%\item<5-> Tres
%\end{itemize}
%\vspace{3cm} 
%\end{column}
%\begin{column}{4cm}
%\begin{overprint}
%\includegraphics<2>[scale=0.05]{images/logo_usc.eps}
%\includegraphics<4>[scale=0.10]{images/logo_usc.eps}
%\includegraphics<6>[scale=0.15]{images/logo_usc.eps}
%\end{overprint}
%\end{column}
%\end{columns}
%\end{frame}
%
%\subsection{Cuando se necesita más espacio} 
%\begin{frame}[plain]
%\frametitle{Pantalla plana con sólo una figura}
%\begin{figure}
%\includegraphics[scale=0.3]{images/logo_usc.eps} 
%\caption{Una figura grande}
%\end{figure}
%\end{frame}

\end{document}

