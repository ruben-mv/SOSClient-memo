%%% Datos del TFG
\newcommand{\tfgauthor}{Rub{\'e}n Mosquera Varela}
\newcommand{\tfgtutor}{Jos{\'e} Ram{\'o}n R{\'i}os Viqueira}
\newcommand{\tfgcotutor}{Manuel Antonio Regueiro Seoane}
\newcommand{\tfgtitle}{Visualización de observacións en SIX de escritorio}
\newcommand{\tfgsubtitle}{}
\newcommand{\tfgtitleshort}{Visualización de observacións en SIX de escritorio} %Para meta-datos PDF
\newcommand{\tfgcopyright}{Copyright (C) 2015, \tfgauthor}
\newcommand{\tfglicense}{Creative Commons (by-sa) 2.5 Spain}
\newcommand{\tfglicenseurl}{http://creativecommons.org/licenses/by-sa/3.0/}
\newcommand{\tfgkeywords}{QGIS SOS OGC Observation}
\newcommand{\logoUni}{images/logo_usc.eps} % Logotipo de la Universidad

%%%
% CONFIG: Meta-datos para inclusión en PDF y XMP
%
\hypersetup{
    pdftitle        = {\tfgtitleshort},
    pdfauthor       = {\tfgauthor},
    pdfsubject      = {\tfgtitle},
    pdfkeywords     = {\tfgkeywords},
    pdfcreator      = {\tfgauthor},
    pdfproducer     = {\tfgauthor~powered~by~\LaTeX},
%    pdfcopyright    = {\tfgcopyright},
%    pdflicenseurl   = {\tfglicenseurl},
%    pdfstartview    = FitH % Ajustar la página a la ventana
}

%----------------------------------------------------------------------------------------
%	FLOATS: TABLES, FIGURES AND CAPTIONS SETUP
%----------------------------------------------------------------------------------------

%RISCOS
\newcounter{riskcnt}
\newcommand{\risktable}[7]{
	\renewcommand{\theriskcnt}{#1}
	\refstepcounter{riskcnt}
	~\\
	\noindent\colorbox{gray!20}{\parbox{\textwidth}{\hspace{0.4cm}\textsc{{#1}.- {#2}}}}\label{rsc:#1}
	\begin{description}[style=multiline,labelindent=0.4cm,leftmargin=4.7cm,itemsep=1pt, partopsep=10pt]
 	\item[Descrición] #3
 	\item[Probabilidade] #4 
 	\item[Impacto] #5 
 	\item[Nivel de exposición] #6 
 	\item[Resposta ao risco] #7 
	\end{description}
	%Se referencia con \ref{rsc:#1}.
	\addcontentsline{lot}{table}{{\theriskcnt} #2}
}
%CASOS DE USO
\newcounter{uccnt}
\newcommand{\uctable}[7]{
	\renewcommand{\theuccnt}{#1}
	\refstepcounter{uccnt}
	~\\
	\noindent\colorbox{gray!20}{\parbox{\textwidth}{\hspace{0.4cm}\textsc{{#1}.- {#2}}}}\label{uc:#1}
	\begin{description}[style=multiline,labelindent=0.4cm,leftmargin=4.5cm,itemsep=1pt, partopsep=10pt]
 	\item[Propósito] #3 
 	\item[Actores] Usuario
 	\item[Relacións] #4 
 	\item[Precondicións] #5 
 	\item[Poscondicións] #6
 	\item[Escenario] #7
	\end{description}
	%Se referencia con \ref{uc:#1}.
	\addcontentsline{lot}{table}{{\theuccnt} #2}
}

%REQUISITOS
\newcounter{reqcnt}
\newcommand{\reqtable}[5]{
	\renewcommand{\thereqcnt}{#1}
	\refstepcounter{reqcnt}
	~\\
	\noindent\colorbox{gray!20}{\parbox{\textwidth}{\hspace{0.4cm}\textsc{{#1}.- {#2}}}}\label{req:#1}
	\begin{description}[style=multiline,labelindent=0.4cm,leftmargin=4.5cm,itemsep=1pt, partopsep=10pt]
 	\item[Descrición] #3 
 	\item[Relevancia] #4
 	\item[Criterio de validación] #5 
	\end{description}
	%Se referencia con \ref{req:#1}.
	\addcontentsline{lot}{table}{{\thereqcnt} #2}
}

%PROBAS
\newcounter{testcnt}
\newcommand{\testtable}[5]{
	\renewcommand{\thetestcnt}{#1}
	\refstepcounter{testcnt}
	~\\
	\noindent\colorbox{gray!20}{\parbox{\textwidth}{\hspace{0.4cm}\textsc{{#1}.- {#2}}}}\label{test:#1}
	\begin{description}[style=multiline,labelindent=0.4cm,leftmargin=4.5cm,itemsep=1pt, partopsep=10pt]
 	\item[Requirimentos relacionados] #3 
 	\item[Descrición] #4
 	\item[Resultado] #5 
	\end{description}
	%Se referencia con \ref{test:#1}.
	\addcontentsline{lot}{table}{{\thetestcnt} #2}
}

%----------------------------------------------------------------------------------------
%	HYPHENATION
%----------------------------------------------------------------------------------------
\hyphenation{In-te-re-se}
\hyphenation{CiTIUS}
\hyphenation{unha}
\hyphenation{SOSClient}
\hyphenation{SOSClientDialog}
\hyphenation{SOSPlot}
\hyphenation{QgsMapToolCaptureSpatialOperand}
\hyphenation{SOSPlotDialog}
\hyphenation{GetObservations}
\hyphenation{GetCapabilities}
\hyphenation{Python}