\chapter{Deseño de software}
O longo deste capítulo detállase a arquitectura do sistema a desenvolver. Mostrase primeiro a arquitectura xeral e a súa interacción con sistemas externos e posteriormente detallase máis polo miúdo a estrutura dos distintos compoñentes do sistema.

\section{Arquitectura do sistema}
Xa na definición dos obxectivos do sistema se identifican dúas partes diferenciadas dentro do sistema. Por un lado a comunicación co servidor SOS, para a obtención dos datos de observacións, e por outro a visualización e explotación destes datos. Esta diferenciación trasládase directamente á arquitectura do sistema, que se divide en dous compoñentes. Na figura \ref{fig:diaComponentes} represéntanse estes dous subsistemas, acompañados polos demais sistemas externos que interactuar con eles. 

Deseño: cómo se realiza o Sistema, a división deste en diferentes compoñentes e a comunicación entre eles. Así mesmo, determinarase o equipamento hardware e software necesario, xustificando a súa elección no caso de que non fora un requisito previo. Debe achegarse a un nivel suficiente de detalle que permita comprender a totalidade da estrutura do produto desenvolvido, utilizando no  posible representacións gráficas.
