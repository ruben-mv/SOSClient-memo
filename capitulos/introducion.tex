\chapter{Introdución}
a)Introdución:  composta  por  Obxectivos  Xerais,  Relación  da  Documentación  que  conforma  a 
Memoria,  Descrición  do  Sistema,  Información  Adicional  de  Interese  (métodos,  técnicas  ou 
arquitecturas utilizadas, xustificación da súa elección, etc.). 
 
\section{Contextualización}
\section{Motivación e Obxectivos}
\section{Estrutura da memoria}

Este documento estrutúrase en 6 capítulos, un apéndice e a bibliografía empregada.\\

Neste capítulo de introdución descríbese a motivación e obxectivos do proxecto e contextualizase o mesmo describindo brevemente os conceptos empregados e a situación actual da técnica nas areas de coñecemento relacionadas.\\

No capítulo 2 abórdanse os aspectos relativos á xestión do proxecto: a definición do alcance, a metodoloxía, planificación temporal e xestión de riscos e da configuración.\\

No capítulo 3 detállase a análise do software, describindo os casos de uso identificados e os requisitos extraidos dos mesmos.\\

No capítulo 4 descríbese o deseño do software, tanto a arquitectura do mesmo como o comportamento de cada unha das compoñentes desde o punto de vista estático e dinámico.\\

No capítulo 5 documéntase en orde cronolóxico o proceso de implementación e as probas realizadas para cumprir os requisitos establecidos.\\

No capítulo 6 expóñense as conclusións extraidas de todo o proxecto e detállanse varias propostas de traballo futuro.\\
%\item[Apéndice A:] \textit{Manual técnico.}

O apéndice A é unha guía para a instalación e manexo básico do \emph{plugin} desenvolvido.\\
