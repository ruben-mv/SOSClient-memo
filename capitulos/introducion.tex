\chapter{Introdución}

Artigo 23. Documentación a Presentar 
A  Memoria  presentada  amosará  o  traballo  desenvolvido  polo  estudante.  Como  norma  xeral,  a 
memoria incorporará a seguinte documentación: 
a)Introdución:  composta  por  Obxectivos  Xerais,  Relación  da  Documentación  que  conforma  a 
Memoria,  Descrición  do  Sistema,  Información  Adicional  de  Interese  (métodos,  técnicas  ou 
arquitecturas utilizadas, xustificación da súa elección, etc.). 
b)Planificación  e  Presupostos:  debe  incluír  a  estimación  do  costo  (presuposto)  e  dos  recursos 
necesarios para efectuar a implantación do Traballo, xunto coa planificación temporal do mesmo e 
a división en fases e tarefas. Recoméndase diferenciar os costos relativos a persoal dos relativos a 
outros gastos como instalacións e equipos. 
c)Especificación de Requisitos: debe indicarse, polo miúdo, a especificación do Sistema, xunto coa 
información  que  este  debe  almacenar  e  as  interfaces  con  outros  Sistemas,  sexan  hardware  ou 
software, e outros requisitos (rendemento, seguridade, etc.). 

5 REGULAMENTO TRABALLO FIN DE GRAO ENXEÑARÍA INFORMÁTICA 
d)Deseño:  cómo  se  realiza  o  Sistema,  a  división  deste  en  diferentes  compoñentes  e  a 
comunicación  entre  eles.  Así  mesmo,  determinarase  o  equipamento  hardware  e  software 
necesario, xustificando a súa elección no caso de que non fora un requisito previo. Debe achegarse 
a  un  nivel  suficiente  de  detalle  que  permita  comprender  a  totalidade  da  estrutura  do  produto 
desenvolvido, utilizando no posible representacións gráficas. 
e)Conclusións e Posibles Ampliacións. 
f)Manuais Técnicos: en función do tipo de Traballo e metodoloxía empregada, o contido poderase 
dividir  en  varios  documentos.  En  todo  caso,  neles  incluirase  toda  a  información  precisa  para 
aquelas persoas que se vaian a encargar do desenvolvemento e/ou modificación do Sistema (por 
exemplo  código  fonte,  recursos  necesarios,  operacións  necesarias  para  modificacións  e  probas, 
posibles problemas, etc.). O código fonte poderase entregar en soporte informático en formatos 
PDF ou postscript. 
g)Manuais  de  Usuario:  incluirán  toda  a  información  precisa  para  aquelas  persoas  que  utilicen  o 
Sistema:  instalación,  utilización,  configuración,  mensaxes  de  erro,  etc.  A  documentación  do 
usuario debe ser autocontida, é dicir, para o seu entendemento o usuario final non debe precisar 
da lectura de outro manual técnico. 
A  Comisión  de  Traballos  Fin  de  Carreira  de  Enxeñaría  Informática  poderá  autorizar  a  realización 
dunha  Memoria  cunha  documentación  diferente  da  mencionada  atendendo  ás  especiais 
características  dun  Traballo  determinado.  Esta  solicitude  deberase  axuntar  ó  Anteproxecto  ou 
presentala antes da data de solicitude de defensa. 
No  caso  de  axuntala  ó  Anteproxecto,  a  Comisión  de  Traballos  Fin  de  Carreira  de  Enxeñaría 
Informática poderá denegar esta solicitude aínda que o Anteproxecto sexa aceptado. 
Artigo 24. Soporte Físico e Instalación da Aplicación 
No caso de que o Traballo implique o desenvolvemento de software, deberase entregar en algún 
tipo de soporte dixital o código fonte, o executable e todos os demais arquivos necesarios para o 
correcto funcionamento da aplicación. 
A aplicación deberase instalar no ordenador que indique a Comisión de Traballos Fin de Carreira 
de  Enxeñaría  Informática.  A  instalación  deberá  estar  dispoñible  e  plenamente  funcional  no 
momento da entrega da documentación. 
 
 
Artigo 25. Entrega da Documentación 
A solicitude de defensa, dirixida á Comisión de Traballos Fin de Carreira de Enxeñaría Informática, 
debe  levar  o  visto  e  prace  dos  Titores  do  Traballo.  O  título  do  Traballo  deberá  especificarse  en 
galego, castelán e inglés. A solicitude irá acompañada de catro copias en papel e catro copias en 
formato dixital da documentación indicada nos Artigos 23 e 24. A dita solicitude entregarase, na 
Secretaría administrativa da Escola, no prazo fixado pola Comisión de Traballos Fin de Carreira de 
Enxeñaría Informática para cada convocatoria (mínimo 7 días antes da data fixada para a lectura). 
Xunto  coa  documentación  do  Traballo,  o  estudante  entregará  unha  petición  de  medios  para  a 
defensa do mesmo, e asinará a correspondente autorización de difusión e a declaración de que se 
trata dun traballo orixinal. 

Introdución: composta por Obxectivos Xerais, Relación da Documentación que conforma a Memoria, Descrición do Sistema, Información Adicional de Interese (métodos, técnicas ou arquitecturas utilizadas, xustificación da súa elección, etc.).
 
\section{Contextualización}
\section{Motivación e Obxectivos}
\section{Estrutura da memoria}
