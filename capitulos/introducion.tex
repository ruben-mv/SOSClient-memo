\chapter{Introdución}
a)Introdución:  composta  por  Obxectivos  Xerais,  Relación  da  Documentación  que  conforma  a 
Memoria,  Descrición  do  Sistema,  Información  Adicional  de  Interese  (métodos,  técnicas  ou 
arquitecturas utilizadas, xustificación da súa elección, etc.). 
 
\section{Contextualización}
\section{Motivación e Obxectivos}
\section{Estrutura da memoria}

Este documento estrutúrase en 6 capítulos, dous apéndices e a bibliografía empregada.

\begin{description}
\item[Capítulo 1:] \textit{Introdución.}
\item[Capítulo 2:] \textit{Xestión do proxecto.} Abórdanse os aspectos relativos á xestión do proxecto: a definición do alcance, a metodoloxía, planificación temporal e xestión de riscos e da configuración.
\item[Capítulo 3:] \textit{Análise de requisitos.} Detallase a análise do software, describindo os casos de uso identificados e os requisitos extraidos dos mesmos.
\item[Capítulo 4:] \textit{Deseño de software.} Descríbese o deseño do software desde a arquitectura do mesmo como o comportamento de cada unha das compoñentes tanto desde o punto de vista estático coma dinámico.
\item[Capítulo 5:] \textit{Implementación e probas.} Documéntase en orde cronolóxico o proceso de implementación e as probas realizadas para cumprir os requisitos establecidos.
\item[Capítulo 6:] \textit{Conclusións e traballo futuro.} Expóñense as conclusións extraidas de todo o proxecto e detállanse varias propostas de traballo futuro.
%\item[Apéndice A:] \textit{Manual técnico.}
\item[Apéndice A:] \textit{Manual de usuario.} Guía para a instalación e manexo básico do \emph{plugin} desenvolvido.
\item[Bibliografía:] Referencias consultadas para o desenvolvemento do proxecto.
\end{description}
