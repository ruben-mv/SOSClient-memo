\chapter{Introdución}
a)Introdución:  composta  por  Obxectivos  Xerais,  Relación  da  Documentación  que  conforma  a 
Memoria,  Descrición  do  Sistema,  Información  Adicional  de  Interese  (métodos,  técnicas  ou 
arquitecturas utilizadas, xustificación da súa elección, etc.). 
 
\section{Contextualización}
\section{Motivación e Obxectivos}
\section{Estrutura da memoria}

Este documento estrutúrase en 6 capítulos, dous apéndices e a bibliografía empregada.

\begin{description}
\item[Capítulo 1:] \textit{Introdución.}
\item[Capítulo 2:] \textit{Xestión do proxecto.} Neste capítulo abórdanse os aspectos relativos á xestión do proxecto: a definición do alcance, a metodoloxía, planificación temporal e xestión de riscos e da configuración.
\item[Capítulo 3:] \textit{Análise de requisitos.}
\item[Capítulo 4:] \textit{Deseño de software.}
\item[Capítulo 5:] \textit{Implementación e probas.}
\item[Capítulo 6:] \textit{Conclusións e traballo futuro.}
\item[Apéndice A:] \textit{Manual técnico.}
\item[Apéndice B:] \textit{Manual de usuario.}
\item[Bibliografía:] Referencias consultadas para o desenvolvemento do proxecto.
\end{description}
