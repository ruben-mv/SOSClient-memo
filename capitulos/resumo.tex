\pagestyle{plain}
\chapter*{Resumo}
No eido da observación medioambiental xéranse a cada instante inxentes cantidades de datos, e a estimación de crecemento a curto prazo é considerable. Este gran volume de datos é tremendamente heteroxéneo, en gran medida pola ampla variedade de dispositivos e sistemas de sensorización empregados. O acceso a toda esta información dende ferramentas de axuda á toma de decisións está, en xeral, dificultado pola falta de interoperabilidade derivada dos diferentes formatos de representación.

Co fin de facilitar o intercambio de datos de observación, o Open Geospatial Consortium (OGC), mediante a súa iniciativa Sensor Web Enablement (SWE), propón a adopción de diferentes estándares de servizos xeoespaciais a través da web. De gran relevancia para este traballo son o estándar Observations and Measurements (O\&M), que define o modelo de datos e a súa codificación XML para a representación de observacións medioambientais, e o estándar Sensor Observation Service (SOS), que define a interface de servizo web para a publicación de todo tipo de información relativa a calquera tipo de sensor e as observacións realizadas polo mesmo.

O nivel de implantación destes estándares é aínda moi baixo, entre outros motivos pola a falta de soporte dos mesmos nos Sistemas de Información Xeográfica (SIX) de propósito xeral.

Así pois, o obxectivo principal deste traballo de fin de grao é dotar ó QGIS da capacidade de consultar diferentes servidores de datos medioambientais a través da interface SOS, e representar os datos obtidos no contorno de mapas propio da ferramenta. Deste xeito permítese a integración de datos de observación de diferente natureza de forma sinxela para a súa posterior exploración e análise por parte do usuario, aproveitando toda a funcionalidade de QGIS, o SIX de escritorio máis empregado e coa comunidade, tanto de usuarios como de desenvolvedores, máis numerosa, dentro do ámbito do software libre.

O \emph{plugin} resultado deste proxecto incorporase ó repositorio oficial de \emph{plugins} da ferramenta QGIS co nome SOS Client.