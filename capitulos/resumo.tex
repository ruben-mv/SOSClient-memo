\pagestyle{plain}
\chapter*{Resumo}
No eido da observación medio ambiental xéranse a cada instante inxentes cantidades de datos, e esta tendencia vai en aumento, non obstante a explotación dos mesmos está limitada por a falta de interoperabilidade entre os distintos sistemas desenvolvidos polas diferentes organizacións involucradas.

Co obxectivo de atallar este problema, o Open Geospatial Consortium (OGC) desenvolve o estándar Sensor Observation Service (SOS) no que define unha interface de servizo web para a publicación de todo tipo de información relativa a calquera tipo de sensor e as observacións realizadas polo mesmo.

O nivel de implantación deste estándar é aínda moi baixo, entre outros motivos pola a falta de soporte do mesmo nos Sistemas de Información Xeográfica (SIX) de propósito xeral.

Así pois, empregando a metodoloxía \emph{Scrum} desenvólvese un \emph{plugin} para o SIX de escritorio de código aberto QGIS para permitir conectarse con servidores que implementen o servizo SOS versión 1.0, de xeito que se podan integrar as observacións descargadas do mesmo no contorno de mapas da ferramenta. Tamén se implementa a funcionalidade de representar as observacións en gráficas en 2D.

O \emph{plugin} obxecto deste proxecto incorporouse ó repositorio oficial de \emph{plugins} da ferramenta QGIS co nome SOS Client.